\section{Introduction}\label{sec:introduction}
Clusters of galaxies form the largest bound objects in the universe, and as such their study is a cornerstone in modern day astronomy. First recognized by 19th century astronomers, their place was solidified when Edwin Hubble proofed their constituent nebulae where not bound to the Milky Way \citep{Hubble1926} but collections of stars similar to the Milky Way. Work to understand their nature and origin began in ernest when Hubble \& Humason (1931) used the virial theorem and the galaxy velocities in the centers of the Virgo (Smith 1936) and Coma (Zwicky 1933) clusters to derive their masses. The immense mass derived exceeded the total stellar mass contributed by all galaxies many times over. This lead Zwicky to theorize the existence of large amounts of non-luminous matter, and coining the term ``dark matter'' (DM), which we still use today.  

Modern astronomy gives the composition of galaxy clusters in three many parts. The galaxies themselves comprise the most obvious feature, and contain a large portion (but not the entirety) of the luminous matter (stars) in the cluster. The intracluster medium (ICM) is the space between the cluster galaxies and is composed many of ordinary matter (baryons) which are super heated to tens of thousands of kelvin. The ICM contains the bulk of the cluster's baryonic matter, and while it is very hot, it is not very dense, with a typical value of $10^{-3}$ particles per cubic centimeter. The majority of the cluster's mass is located in the DM halo which surrounds the cluster. 

Thought to form out of the primordial density fluctuations in the very early universe, the investigation of their formation and growth began in the 1960s. Quickly, the hierarchical model of structure formation (Press \& Schechter 1974, Gott \& Rees 1975, White \& Rees 1978) where the first stars and stellar clumps grew first which subsequently merged together with dark matter and other gas clumps to form the first galaxies which then continued to merge and grow into the clusters and large scale structures we see today. This accretion of smaller systems is thought to be driven by the gravity of the DM associated with the cluster. Of course, many complicated astrophysical processes are at work during cluster growth and similarly complicated theoretical models seek to explain these processes. For a detailed review of cluster formation see \cite{Kravtsov2012}.

Galaxy clusters stand at the intersection of cosmology and astrophysics. The number and distribution of galaxy clusters across the sky are the finger print of the cosmology imprinted on the universe at its birth. To uncover the underlying cosmology a detailed understanding of the astrophysical processes that describe the motion of constituent galaxies and their impact on the ICM is required.

\subsection{Cluster Cosmology}
While galaxy clusters are only one of many possible cosmological probes (CMB, supernova, BAO, and others), they are important as they trace the peaks in the universal matter density. These peaks are often referred to as the power spectrum of matter density fluctuations or the matter power spectrum. \editorial{start here} Simple inflationary theories predict that just after inflation the matter power spectrum would have been a
simple power law (right hand panel, Figure 1). To understand the formation of the peak, we have to
consider how different sized clumps of matter become clumpier with time, at different stages in the
Universe.
Before matter-radiation equality, the energy in radiation was so great that it provided the dominant
gravitational force. The matter distribution then tends to follow the radiation distribution through the
gravitational interaction. Clumps of radiation pulled themselves together through gravitational self attraction,
making the radiation more clustered. However, radiation also has significant pressure, so small
dense clumps were pushed apart again by the radiation pressure. Once sufficiently diffuse, the
gravitational attraction won over, and the clumps collapsed, rebounded, collapsed etc, oscillating. This is
referred to as acoustic oscillations of the photon-baryon fluid (since the baryons were tightly coupled to the photons through electron scattering and electrostatic attraction of nuclei and electrons).The largest clumps
were too large for radiation pressure to be able to cross, and continued to collapse under gravity. 

Thus clumps on the smallest scales oscillate and, averaging over time, do not collapse significantly under
gravity. Whereas clumps on large scales simply collapse under gravity. The boundary between these two
regimes occurs when the sound speed crossing time is equal to the gravitational infall time, and is referred
to as the Jeans length. (The sound speed tells us how fast radiation pressure can act.) This argument is
exactly the Jeans wavelength argument that you may have seen before. Thus perturbations on scales
larger than the Jeans length grow, and perturbations on scales smaller than the Jeans length do not.
What impact does this have on the matter power spectrum? When perturbations on a given scale grow,
the Universe becomes more clumpy on those scales. Fourier theory says that we can represent the
distribution of matter as the sum of sine waves. Therefore when the distribution of matter becomes more
clumpy, this is equivalent to the amplitude of the sine waves growing and therefore the power spectrum
increasing in height. Thus large scales grow and small scales stay constant. Imagine taking the postinflation
power spectrum in the right hand panel of Fig 1 and moving the whole plot upwards, except on
small scales. This starts to induce a bump at small scales.
As time goes on, larger scales are able to oscillate, since there is more time for radiation pressure waves
to travel further. This is equivalent to saying that the Jeans length increases. Thus the left hand side of the
power spectrum continues to increase in height and the bump moves to the left (larger scales).
After matter-radiation equality, the radiation is no longer the dominant component determining the
dynamics. Matter is attracted under gravity, but does not have significant pressure, since it is mostly
assumed to be dark matter. Therefore there are no more acoustic oscillations. All scales clump together
more and more. Thus the whole of the matter power spectrum increases the same on all scales. The
position of the bump is therefore frozen into the shape of the matter power spectrum.
The peak position corresponds to the Jeans length at matter-radiation equality.
Now test your understanding: If there is more matter in the Universe, would you expect the peak in the
matter power spectrum to be further to the left or further to the right? Does this correspond to smaller or
larger scales?
The position of the peak in the matter power spectrum can tell us the amount of matter in the Universe,
often parameterized by Ωm. However, the small scale slope (right hand side of the plot) can also tell us
about the speed at which the dark matter particles are moving. Qualitatively, the faster the dark matter
particles are moving, the more small scale fluctuations are erased. This is because if an overdensity of
matter contains many hot dark matter particles then these particles will quickly travel out of the overdensity
and therefore the overdensity will evaporate. Therefore, if all of the dark matter were hot dark matter then
the matter power spectrum would fall off sharply to zero to the right of the peak.
Observationally it is very hard to measure the matter power spectrum, since we believe that 90 per cent of
the matter is dark and therefore does not emit light. However we can see where the galaxies are and 
calculate the power spectrum of the galaxy distribution. We could then assume that the galaxies ‘trace’ the
dark matter: that is, that they distributed so that where there is a clump of dark matter there is also a
collection of galaxies. An extreme version of this assumption is to say that the galaxy power spectrum is
exactly the same as the matter power spectrum. In fact, we can already measure the power spectra for
different types of galaxies (e.g. plot the spiral galaxy power spectrum and compare with the power
spectrum of elliptical galaxies). We find that power spectra for different types of galaxies have different
amplitudes but very similar shapes. Therefore cosmologists usually make the slightly less restrictive
assumption that the galaxy power spectrum is the same shape as the matter power spectrum, but may
have a different overall height. 



Historical bits of cosmology.
$\sigma_8$ and $\Omega_m$
Principal challenge to precision cosmology

\subsection{State of Play}
Simulations
X-ray Surveys
Weak lensing esimates
SZ surveys
optical estimates
Current Constraints

\subsection{Cluster Surveys in the near-future}
X-ray (eROSITA)
SZ (ACT, SPT)
optical (LSST, DES, HETDEX)