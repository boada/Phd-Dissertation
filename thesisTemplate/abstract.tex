%%%%%%%%%%%%%%%%%%%%%%%%%%%%%%%%%%%%%%%%%%%%%%%%%%%
%
%  New template code for TAMU Theses and Dissertations starting Fall 2012.  
%  For more info about this template or the 
%  TAMU LaTeX User's Group, see http://www.howdy.me/.
%
%  Author: Wendy Lynn Turner 
%	 Version 1.0 
%  Last updated 8/5/2012
%
%%%%%%%%%%%%%%%%%%%%%%%%%%%%%%%%%%%%%%%%%%%%%%%%%%%
%%%%%%%%%%%%%%%%%%%%%%%%%%%%%%%%%%%%%%%%%%%%%%%%%%%%%%%%%%%%%%%%%%%%%
%%                           ABSTRACT 
%%%%%%%%%%%%%%%%%%%%%%%%%%%%%%%%%%%%%%%%%%%%%%%%%%%%%%%%%%%%%%%%%%%%%

\chapter*{ABSTRACT}
\addcontentsline{toc}{chapter}{ABSTRACT} % Needs to be set to part, so the TOC doesnt add 'CHAPTER ' prefix in the TOC.

\pagestyle{plain} % No headers, just page numbers
\pagenumbering{roman} % Roman numerals
\setcounter{page}{2}

\indent The distribution of massive clusters of galaxies depends strongly on the total cosmic mass density, the mass variance, and the dark energy equation of state. As such, measures of galaxy clusters can provide constraints on these parameters and even test models of gravity, but only if observations of clusters can lead to accurate estimates of their total masses. Here, we carry out a study to investigate the ability of a blind spectroscopic survey to recover accurate galaxy cluster masses through their velocity dispersion using probability based and machine learning methods. We focus on the Hobby Eberly Telescope Dark Energy Experiment (HETDEX), which will employ new Visible Integral-Field Replicable Unit Spectrographs (VIRUS), over 420 \degsq\ on the sky with a 1/4.5 fill factor. VIRUS covers the blue/optical portion of the spectrum ($3500-5500~\AAA$), allowing surveys to measure redshifts for a large sample of galaxies out to $z < 0.5$ based on their absorption features or \hbox{[\ion{O}{ii}]} $\lambda$3727 emission (and Lyman-$\alpha$ over $1.9 < z < 3.5$). We use a detailed mock galaxy catalog from a semi-analytic model to simulate surveys observed with VIRUS, including: (1) a blind, HETDEX-like survey with an incomplete but uniform spectroscopic selection function; and (2) IFU surveys that target clusters directly, obtaining spectra of all galaxies in a VIRUS-sized field. For both surveys, we include realistic uncertainties from galaxy magnitude and line-flux limits. We benchmark both surveys against spectroscopic observations with ``perfect" knowledge of galaxy line-of-sight velocities. With HETDEX, we can recover cluster masses to $\sim0.1$ dex which can be further improved to $<0.1$ dex with targeted follow-up observations. This level of cluster mass recovery enables constraints on $\sigma_8$ to $<20$\%, and the unique properties of the observations will provide important calibrations for the optical richness-cluster mass relation. As a follow-up, we present integral field spectroscopy of ten galaxy clusters optically selected from the Sloan Digital Sky Survey's DR8 at $z=0.2-0.3$. We use the Mitchell Spectrograph to measure spectroscopic redshifts and line-of-sight velocities of the galaxies in and around each cluster, determine cluster membership and derive line-of-sight velocity dispersions. We use a traditional power law scaling relation between line-of-sight velocity dispersion and cluster mass and a machine learning based approach to infer a dynamical mass for each cluster. Eight of the clusters are rich ($\lambda>60$) systems with total inferred masses $(1.58-17.37) \times 10^{14}$ \Msol\ ($M_{200c}$), and two are poor ($\lambda<15$) systems with inferred total masses $\sim0.5 \times 10^{14}$ \Msol\ ($M_{200c}$). After comparing to the literature  we use these independent cluster mass estimations to estimate the absolute cluster mass scale, and overall scatter in the optical richness-mass relationship. We find good agreement with recent studies for both the scale and scatter in the relationship. This demonstrates the power of blind spectroscopic surveys such as the forthcoming Hobby Eberly Dark Energy Experiment to provide robust mass estimates which can aid in the the determination of cosmological parameters and help to calibrate the observable-mass relation for future photometric large area-sky surveys.
\pagebreak{}
