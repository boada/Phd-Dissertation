%%%%%%%%%%%%%%%%%%%%%%%%%%%%%%%%%%%%%%%%%%%%%%%%%%%
%
%  New template code for TAMU Theses and Dissertations starting Fall 2012.  
%  For more info about this template or the 
%  TAMU LaTeX User's Group, see http://www.howdy.me/.
%
%  Author: Wendy Lynn Turner 
%	 Version 1.0 
%  Last updated 8/5/2012
%
%%%%%%%%%%%%%%%%%%%%%%%%%%%%%%%%%%%%%%%%%%%%%%%%%%%
%%%%%%%%%%%%%%%%%%%%%%%%%%%%%%%%%%%%%%%%%%%%%%%%%%%%%%%%%%%%%%%%%%%%%%
%%                           SECTION III
%%%%%%%%%%%%%%%%%%%%%%%%%%%%%%%%%%%%%%%%%%%%%%%%%%%%%%%%%%%%%%%%%%%%%



% \renewcommand*{\thefootnote}{\fnsymbol{footnote}}
\chapter[\uppercase{Future Work}]{\uppercase{Future Work}}
% \renewcommand*{\thefootnote}{\arabic{footnote}}
% \setcounter{footnote}{0}

This thesis represents the culmination of the work done over the past four years, and this final chapter summarizes the lessons learned. Over the coming years, the amount of multi-band imaging of the night sky will grow tremendously. The study of galaxy clusters will certainly benefit from the large array of data, but only if we are able to combine datasets in a meaningful way. This combination will push the boundaries of our knowledge of cluster physics, clusters in their role as cosmological probes, and the usefulness of advanced data analysis techniques. It is the author's hope that this thesis has contributed to this push forward. 

We begin by a summary of the two studies presented here. We, then, present a small discuss of different projects possible with a survey like HETDEX, and close with a few questions which remain under active investigation.

\section{Summary} 
The work presented here is a study of how a blind spectroscopic survey, such as HETDEX, will be able to recover accurate cluster masses for galaxy clusters below redshift 0.5. We began with a set of simulated observations to explore this question. Our main results from Chapter~\ref{Chapter1} are as follows:

\begin{itemize}
	\item The blind spectroscopic observations of HETDEX are capable of accurately estimating galaxy cluster masses over the range of redshifts and cluster masses covered by this study. We find that using a machine learning approach where several cluster observables are combined, we estimate cluster masses to a similar level of precision as a fully targeted survey. This is particularly true for high mass clusters. With targeted followup observations, we can improve the mass estimates of all clusters, but the mass bias can be significantly improved for clusters below $10^{13}$ \Msol.
	
	\item The large area surveyed by HETDEX results in a large number of clusters, combined with the sparse sampling of the sky, prevents HETDEX from being statistically limited in its estimation of cosmological parameters. However, the level of precision in our cluster mass estimates will provide an important, independent check on the estimates of $\sigma_8$ and $\Omega_m$ from other studies.
	
	\item Once completed, HETDEX will be able to measure the intrinsic scatter in the optical richness-cluster mass relationship for a wide range of cluster masses. The individual cluster mass estimates will remain unparalleled for such a large number and mass range, providing a much needed understanding of the richness-mass relation. This will both improve existing cosmological parameters estimates, which rely on the optical richness as a cluster mass proxy, and provide accurate calibrations for future large-area sky surveys.  
\end{itemize}

As a followup to the first work, we observe ten galaxy clusters selected from the SDSS to be high mass, and have corresponding richness measurements. Our goal is to provide a practical test to the methods and results presented in the first chapter. Our main results from Chapter 2 are as follows:

\begin{itemize}
	\item We observe and and provide cluster mass estimates for ten clusters, covering two distinct richness regimes, selected from the SDSS. Using the traditional power law estimator, we find they range in mass from $(0.52-17.37) \times 10^{14}$ \Msol\ ($M_{200c}$). The majority of these mass estimates correspond well to previously reported estimates using targeted spectroscopy or X-ray observations.
	
	\item The machine learning approach to cluster mass estimation, while powerful, is only appropriate when a deep understanding of the data used to train the method is available. This deep understanding is important as to not bias the mass predictions from the machine learning method. While, in this work, the training sample was not appropriate for the clusters observed, the machine learning approach remains highly promising, and potentially a revolutionary tool for modern astronomy.
	
	\item Using the ten clusters observed, we measure the absolute cluster mass scale and intrinsic scatter of the optical richness-cluster mass relationship. We find an overall mass scale which broadly agrees with previously reported values, although the uncertainties associated with our recovered value are large due to the relatively low number of clusters included and the uncertainties on the individual cluster mass estimates. We also measure an intrinsic scatter similar to the value reported by other studies. In both cases, this is very promising for the upcoming HETDEX survey's ability to place extremely narrow constraints on these very important systematics.
\end{itemize}

\section{Other Potential Investigations}\label{sec: future}
\subsection{Investigation of Cluster Miscentering}
Misidentification of the cluster center (due to the mass distribution not being directly observable) can add significant error to the estimate of its mass, and remain a significant challenge for optical mass estimates \citeeg{Duarte2015}. Fortunately, hierarchical growth suggests the brightest cluster galaxy (BCG) should be located at the center of the cluster's potential well. Unfortunately, this is not always the case \citeeg{Skibba2011}. When there is no clear central galaxy or the brightest (most massive) galaxy is difficult to identify, the centroid of the hot intracluster medium or a weighted centroid of the member galaxy positions \citeeg{George2012} can be used.

Many optical, photometry-based cluster finding algorithms (\eg, redMapper) assume a massive cD galaxy is present in each cluster, which gives a success rate of approximately 85\% and is described as the ``least bad'' option for identifying the cluster center. The remaining 15\% of clusters often lack a defined center because the cluster's BCG does not match the expected color of a massive, elliptical galaxy. Specifically, cool-core clusters can have blue, star-forming central galaxies, confusing the cluster center determination.

Star-forming central galaxies are most common in clusters at the lowest end of the halo mass function, which coincide with the majority of clusters in the universe, or in clusters with increasing redshift (the Butcher-Oemler effect). To help correct this issue, targeted observations with HETDEX's VIRUS instrument could potentially identify the center of motion through the observed dynamics of the member galaxies. This is insensitive to the type of central galaxy and could provide additional clues to optically select central galaxies which do not match the predicted color. Given a large enough sample, a statistical approach will be very powerful.

\subsection{The Search for Clusters above $z\sim1$}
The identification of galaxy clusters in the low-$z$ universe through optical imaging often relies on a well defined red sequence of cluster galaxies (\eg, RedMapper). Beyond $z=1$ the this becomes increasingly hard due to the depth required and the increasing lack of well defined red sequence. Therefore, new methods to search for and identify clusters in the early universe with optical/NIR imaging must be adopted.
 
The search for clusters in point data (\eg, RA, DEC, $z$) is not unique to astronomy and, as such, many methods (k-means, mixture models, etc.) have been developed to tackle the problem. \cite{Dehghan2014} used one such data agnostic machine learning algorithm, the Density-Based Spatial Clustering of Applications with Noise (DBSCAN; \citealt{Ester1996}), to search for clusters, groups, and filaments in the Extended Chandra Deep Field South. They successfully identified structures at $z<1$ using a ground based spectroscopic survey and the MUSYC photometric catalog. The HETDEX spectroscopic data could provide improved distance estimates for many potential galaxy clusters, and combined with modern clustering algorithms, such as Ordering Points To Identify the Clustering Structure (OPTICS; \citealt{Ankerst1999}) or a hierarchical version of DBSCAN (HDBSCAN; \citealt{Campello2013}), which are designed to address the weaknesses of DBSCAN, could provide important insights into the optimization of cluster finding at high redshift. Such advanced clustering searches can also be extended to the large sky-area of PanSTARRS to search for dwarf galaxies or other stellar associations and will eventually be useful to the large-area survey of WFIRST-AFTA. Ultimately, such a study could aid in the development and application of data agnostic machine learning to astronomy, provide improved cluster detection at all redshifts, and suggest potential followup targets requiring the \emph{James Webb Space Telescope}.

\subsection{Outstanding Questions}
The work presented here is only a small piece in the effort to understand how a blind spectroscopic survey could impact the role of galaxy clusters as cosmological probes. As with any study, outstanding questions remain. Here are two potential followup investigations:

\begin{itemize}
	\item A deep study and optimization of the machine learning methods used to predict cluster masses. There are many machine learning algorithms which are capable of combining many cluster observables to predict a cluster mass. In addition, each method has a series of hyper-parameters which can greatly impact the method's performance from one data set to another, and from one potential problem to another. A deep investigation of an appropriate way to combine observational parameters, which parameters to combine, could provide an enormous amount of insight into the best follow up observations, and greatly improve predicted cluster masses. \cite{Acquaviva2016} conducted a similar study for the estimation of galaxy metallicity. A similar study for HETDEX applications would be extremely valuable.
	\item Impact of cluster contamination on the machine learning predicted masses. In our simulated HETDEX survey, we assume that we have knowledge of a cluster's member galaxies, and so this initial work is not effected by the presence of interloping galaxies. A study of how the sparse spatial sampling, and the presence of interloping galaxies \citeeg{Ntampaka2015a} could be mitigated by machine learning, or another method, is extremely important. Such a study will be required to estimate the best possible cluster masses when HETDEX begins taking data. 
\end{itemize} 

Of course, there is also the deep study of the physics surrounding the clusters themselves, through the study of the ICM \citeeg{Mantz2015}, any active galactic nuclei discovered \citeeg{Kirk2015, Sobral2015}, correlations with other observations \citeeg{Pipino2010}, tests for substructure \citeeg{Yu2015}, and many others. 

We provide all of the code used to conduct this study at https://github.com/boada. Large file size prevents including the source data with the analysis routines. The authors are happy to provide them, if requested.
