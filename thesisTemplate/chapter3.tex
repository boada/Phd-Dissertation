%%%%%%%%%%%%%%%%%%%%%%%%%%%%%%%%%%%%%%%%%%%%%%%%%%%
%
%  New template code for TAMU Theses and Dissertations starting Fall 2012.  
%  For more info about this template or the 
%  TAMU LaTeX User's Group, see http://www.howdy.me/.
%
%  Author: Wendy Lynn Turner 
%	 Version 1.0 
%  Last updated 8/5/2012
%
%%%%%%%%%%%%%%%%%%%%%%%%%%%%%%%%%%%%%%%%%%%%%%%%%%%
%%%%%%%%%%%%%%%%%%%%%%%%%%%%%%%%%%%%%%%%%%%%%%%%%%%%%%%%%%%%%%%%%%%%%%
%%                           SECTION III
%%%%%%%%%%%%%%%%%%%%%%%%%%%%%%%%%%%%%%%%%%%%%%%%%%%%%%%%%%%%%%%%%%%%%



% \renewcommand*{\thefootnote}{\fnsymbol{footnote}}
\chapter[\uppercase{Future Work}]{\uppercase{Future Work}}
% \renewcommand*{\thefootnote}{\arabic{footnote}}
% \setcounter{footnote}{0}

\section{Introduction} Clusters of galaxies form the largest bound objects in the universe, and as such their study is a cornerstone in modern day astronomy. Thought to form out of the primordial density fluctuations in the very early universe, the number and distribution of galaxy clusters across the sky is the finger print of the cosmology imprinted on the universe at its birth. The $\Lambda$CDM model of cosmology makes explicit predictions about the number and masses of galaxy clusters throughout the universe. Connecting these predictions to a set of, sufficiently large in size, observed clusters remains a principal problem. Specifically, the largest threat to modern, precision, cluster cosmology is not the identification of large numbers of clusters (the total number of clusters known is only going up) but the accurate recovery of galaxy cluster mass \citeeg{Sehgal2011,Plank2014, Bocquet2015}. 

As mass is not a direct observable, a lot of work is underway to characterize galaxy cluster masses with an observable feature of galaxy clusters. Observed X-ray temperatures and luminosities correlate tightly with a cluster's dynamical mass \citeeg{Mantz2010, Rykoff2014}, especially for dynamically relaxed clusters \citeeg{Mantz2015}. The Sunyaev--Zel'dovich effect (SZE; \citealt{Sunyaev1972}), which uses the up--scattering of cosmic microwave background (CMB) photons to estimate cluster masses, provides accurate estimations of mass \citeeg{Vanderlinde2010, Sehgal2011}, but the ability to detect low mass galaxy clusters is currently limited by technology \citeeg{Carlstrom2002a}. Optical studies \citeeg{Rozo2010, Rozo2015a} primarily use the richness \citeeg{Abell1958,Rykoff2012} or galaxy velocity dispersions to estimate the mass, and are often used to calibrate other mass estimators \citeeg{Ruel2014, Sifon2015}. 

Today, the greatest number of clusters are being discovered using large SZE surveys with the South Pole Telescope (SPT; \citealt{Carlstrom2011}) or the Atacama Cosmology Telescope (ACT; \citealt{Swetz2011}). However, deep, wide field optical surveys, such as the Dark Energy Survey (DES; \citealt{DES2005}) will discover many more, low mass clusters in the near future. Such clusters will rely on spectroscopic follow up to better constrain their dynamical mass. But, as the number of clusters grows to many tens of thousands, spectroscopic followup becomes unfeasible. Therefore, large, single telescope, spectroscopic surveys will be required to reduce systematics and calibrate the observable--mass relation to a level that will allow accurate mass estimations using other methods.

In this work, we present a pilot study of ten massive galaxy clusters using integral field spectroscopy with the Mitchell Spectrograph as a pilot program for the Hobby Eberly Dark Energy Experiment (HETDEX; \citealt{Hill2008}) survey. HETDEX is a forthcoming blind spectroscopic survey that could potentially be used to accurately calibrate the observable--mass relation for a significant number of galaxy clusters at both extremes of the mass scale. At present, because HETDEX is designed to measure the dark energy equation of state at $z\sim2$, the applicability to galaxy cluster science has not yet been investigated. We began this investigation with \defcitealias{Boada2016}{Paper I}\cite{Boada2016} (hereafter \citetalias{Boada2016}). The second installment of this two part work, the goal of this study is to obtain spectroscopic redshifts of the individual cluster galaxies, determine the velocity dispersion and to infer each cluster's dynamical mass. This allows us to compare the inferred mass with other mass estimators (\eg, the clusters in this sample have deep \textit{Chandra} or \textit{XMM-Newton} X-ray data, and richness measurements) with the aim of reducing the scatter in the richness--mass relation, $\sigma_{M|\lambda}$. The ability of HETDEX to further constrain optically derived masses is of paramount importance to upcoming large photometric surveys. This study provides insight into how well a HETDEX type survey will constrain mass estimations and cosmological parameters in the future.

The layout of this work is the following. In Section~\ref{sec:design} we discuss the target selection and the setup of the MS used to conduct the observations. Section~\ref{sec:data reduction} describes the methods and tools used to reduce the observations. We present our redshift catalog, cluster members and cluster dynamical properties in Section~\ref{sec:analysis}. In Section~\ref{sec:discussion}, we compare and discuss the different mass estimations and remark on the applicability of these methods for HETDEX. Finally, we summarize this work in Section~\ref{sec:summary}.

Throughout this paper, we use a concordance cosmological model ($\Omega_\Lambda = 0.7$, $\Omega_m = 0.3$, and $H_0= 70$ \kms \mpc), assume a Chabrier initial mass function \citep{Chabrier2003}, and use AB magnitudes \citep{Oke1974} unless specifically noted.
