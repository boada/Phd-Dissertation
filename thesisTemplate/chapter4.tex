%%%%%%%%%%%%%%%%%%%%%%%%%%%%%%%%%%%%%%%%%%%%%%%%%%%
%
%  New template code for TAMU Theses and Dissertations starting Fall 2012.  
%  For more info about this template or the 
%  TAMU LaTeX User's Group, see http://www.howdy.me/.
%
%  Author: Wendy Lynn Turner 
%	 Version 1.0 
%  Last updated 8/5/2012
%
%%%%%%%%%%%%%%%%%%%%%%%%%%%%%%%%%%%%%%%%%%%%%%%%%%%
%%%%%%%%%%%%%%%%%%%%%%%%%%%%%%%%%%%%%%%%%%%%%%%%%%%%%%%%%%%%%%%%%%%%%%
%%                           SECTION III
%%%%%%%%%%%%%%%%%%%%%%%%%%%%%%%%%%%%%%%%%%%%%%%%%%%%%%%%%%%%%%%%%%%%%



% \renewcommand*{\thefootnote}{\fnsymbol{footnote}}
\chapter[\uppercase{Summary and Future Work}]{\uppercase{Summary and Future Work}}
% \renewcommand*{\thefootnote}{\arabic{footnote}}
% \setcounter{footnote}{0}

This thesis represents the culmination of the work done over the past four years, and this final chapter summarizes the lessons learned. Over the coming years, the amount of multi-band imaging of the night sky will grow tremendously. The study of galaxy clusters will certainly benefit from the large array of data, but only if we are able to combine datasets in a meaningful way. This combination will push the boundaries of our knowledge of cluster physics, clusters in their role as cosmological probes, and the usefulness of advanced data analysis techniques. It is the author's hope that this thesis has contributed to this push forward. 

We begin by a summary of the two studies presented here. We, then, present a small discuss of different projects possible with a survey like HETDEX, and close with a few questions which remain under active investigation.

\section{Summary}\label{sec:summary} 
In Chapter 2, we present detailed simulations of the upcoming HETDEX survey's applicability to the detection and total mass measurement of galaxy clusters. Using mock galaxy catalogs and HETDEX-like observational strategies and limits, we observe our simulated sky, estimate the number of clusters observed, and derive basic cluster parameters, redshift, line-of-sight velocity dispersion (LOSVD). Using a traditional power law-based, velocity dispersion, scaling relation along with more advanced probability and machine learning (ML) techniques, we estimate each cluster's total mass. We discuss each cluster mass estimate's precision, and discuss HETDEX's ability to constrain the cosmological parameter $\sigma_8 h^{-1}$ based on those predicted cluster masses. In addition, we comment on how HETDEX may improve current and future photometric large-area sky surveys' cluster mass estimates derived from optical richness.

Our main conclusions are the following:
\begin{enumerate}
	\item After considering HETDEX's observational limits, we find 14,189 clusters with at least five cluster members in the HETDEX survey volume. Of those, 1,760 clusters are detected with HETDEX-like survey observations. The number of cluster members recovered with Survey observations is almost exactly 4.5 times fewer than a fully Targeted survey, across both a wide range of redshifts and cluster masses.
	
	\item We find a traditional power law conversion from LOSVD to cluster mass predicts the true cluster mass with little bias or scatter for clusters Log M/\Msol\ $=14.5$ and above. Below this mass the bias and scatter rapidly increases. In contrast, the probability based and ML based cluster mass estimators are able to predict cluster mass with similar or smaller scatter across all cluster masses. The scatter further decreases when the probability based and ML estimators are combined with other cluster observables besides the LOSVD. For HETDEX-like observations and clusters with $13 < \mathrm{Log}\, M/\Msol <14.5$, we find the $\mathrm{ML}_{\sigma, z, N_{gal}}$ method results in the smallest scatter. Below Log M/\Msol\ $=13$ no method with Survey observations gives a bias of less than 50\%. For the highest mass clusters the power law method gives the lowest bias and scatter. In short, no single method is superior in all regards. The technique should be chosen to minimize the desired systematic, but we find $\mathrm{ML}_{\sigma, z, N_{gal}}$ provides the best performance across the large range of cluster masses, and observation strategies.
	
	\item In general, we find that the measured scatter of cluster masses decreases when considering Targeted versus Survey observations. Clusters at all masses can benefit from targeted follow-up observations, although the accuracy gain will be smaller than can be achieved from cluster mass prediction method changes. Targeted follow-up observations reduces the measured scatter by $\sim10\%$ when comparing like recovery methods.
	
	\item The $\sim51\%$ cluster mass accuracy of Survey observations places approximately a 20\% constraint on $\sigma_8 h^{-1}$. This can be tightened to approximately 12\% with follow-up targeted observations. Most importantly, the observations from HETDEX will provide systematics checks on other studies, ultimately improving all future measurements of $\sigma_8 h^{-1}$
	
	\item HETDEX will be able to place important, independent constraints on the amount of scatter in the optical richness-mass relationship. It will to a less extent constrain the overall normalization of the relation. This should provide an important tool in the calibration of large-area sky surveys which rely on photometric data only to estimate cluster masses.
\end{enumerate}

In Chapter 3, we carry out a proof-of-concept study where we present integral field unit observations with the Mitchelle Spectrograph of ten intermediate redshift ($z=0.2-0.3$) galaxy clusters. We observe cluster member galaxies within $R\sim0.5$ Mpc of each cluster, and determine each cluster's membership based on the line-of-sight velocity of each galaxy. The mass of each cluster is determined through a traditional PL and through a machine learning based approach. We use these estimates of cluster mass to measure the absolute mass scale and intrinsic scatter of the optical richness-cluster mass relationship. The goal of this study is to investigate how a blind spectroscopic survey, such as the forthcoming HETDEX, will be able to predict cluster masses, and then to use those masses to help calibrate other observable-cluster mass scaling relations.
Our main results are as follows:
\begin{itemize}
	\item Using a PL scaling relation between the measured line-of-sight velocity dispersion and cluster mass, our low richness ($\lambda < 15$) sample of galaxy clusters have total inferred masses $\sim 0.5 \times 10^{14}$ \Msol\ ($M_{200c}$), and the high richness ($\lambda > 60$) cluster sample has total masses $(1.58-17.37) \times 10^{14}$ \Msol\ ($M_{200c}$). The majority of these estimate are consistent with other published total mass estimates which use a variety of estimation techniques. 
	
	\item The machine learning based approach of galaxy cluster mass estimation, while powerful, requires a deep understanding of both the machine learning algorithms available, training sets that accurately represent the properties of the data, and expert knowledge of the problem domain. To estimate the total masses of the galaxy clusters in our sample using machine learning methods, the native Buzzard catalogs do not provide a suitable training set due to the limited cosmological volume simulated at low redshift. For the redshift range of our cluster sample, the Buzzard catalogs lack similar high line-of-sight velocity dispersion and high mass clusters. This leads to a severe down-weighting of high mass galaxy clusters when the cluster redshift is included as a training feature. When this feature is removed, the machine learning estimated cluster masses improve but are still underestimated due to too few very high mass clusters being included in the simulated volume. A suite of training data drawn from a similarly sized cosmological volume is critical to the reliable prediction of cluster masses. When this is available, the machine learning predicted cluster masses show less bias and lower scatter compared to a traditional power law scaling relation based on velocity dispersion alone.

	\item We fit a optical richness-cluster mass relation to the eight high richness ($\lambda > 60$) clusters. This gives:
		\begin{equation}
			\mathrm{Log}\,M_{200c}/\Msol=1.25\pm{0.78}\, \mathrm{Log}\,\lambda + 12.29\pm{1.68}.  
		\end{equation}
We are unable to place tight constraints on the overall estimate of the normalization due to the relatively few clusters included. We do estimate the scatter in cluster mass at fixed richness, $\sigma_{M|\lambda} \sim 0.25$. This estimate of scatter compares well with other recent studies of the richness-mass relation. This suggests that a blind spectroscopic survey such as HETDEX will be able to provide a crucial, independent measurement of this scatter, to a much high precision than is possible in this work. This bodes well for the success of HETDEX as not only a high redshift galaxy survey but as a important calibration tool for understanding the uncertainties associated with galaxy cluster mass estimates. 
\end{itemize}

\section{Other Potential Investigations}\label{sec: future}
\subsection{Investigation of Cluster Miscentering}
Misidentification of the cluster center (due to the mass distribution not being directly observable) can add significant error to the estimate of its mass, and remain a significant challenge for optical mass estimates \citeeg{Duarte2015}. Fortunately, hierarchical growth suggests the brightest cluster galaxy (BCG) should be located at the center of the cluster's potential well. Unfortunately, this is not always the case \citeeg{Skibba2011}. When there is no clear central galaxy or the brightest (most massive) galaxy is difficult to identify, the centroid of the hot intracluster medium or a weighted centroid of the member galaxy positions \citeeg{George2012} can be used.

Many optical, photometry-based cluster finding algorithms (\eg, redMaPPer) assume a massive cD galaxy is present in each cluster, which gives a success rate of approximately 85\% and is described as the ``least bad'' option for identifying the cluster center. The remaining 15\% of clusters often lack a defined center because the cluster's BCG does not match the expected color of a massive, elliptical galaxy. Specifically, cool-core clusters can have blue, star-forming central galaxies, confusing the cluster center determination.

Star-forming central galaxies are most common in clusters at the lowest end of the halo mass function, which coincide with the majority of clusters in the universe, or in clusters with increasing redshift (the Butcher-Oemler effect). To help correct this issue, targeted observations with HETDEX's VIRUS instrument could potentially identify the center of motion through the observed dynamics of the member galaxies. This is insensitive to the type of central galaxy and could provide additional clues to optically select central galaxies which do not match the predicted color. Given a large enough sample, a statistical approach will be very powerful.

\subsection{The Search for Clusters above $z\sim1$}
The identification of galaxy clusters in the low-$z$ universe through optical imaging often relies on a well defined red sequence of cluster galaxies (\eg, redMaPPer). Beyond $z=1$ the this becomes increasingly hard due to the depth required and the increasing lack of well defined red sequence. Therefore, new methods to search for and identify clusters in the early universe with optical/NIR imaging must be adopted.
 
The search for clusters in point data (\eg, RA, DEC, $z$) is not unique to astronomy and, as such, many methods (k-means, mixture models, etc.) have been developed to tackle the problem. \cite{Dehghan2014} used one such data agnostic machine learning algorithm, the Density-Based Spatial Clustering of Applications with Noise (DBSCAN; \citealt{Ester1996}), to search for clusters, groups, and filaments in the Extended Chandra Deep Field South. They successfully identified structures at $z<1$ using a ground based spectroscopic survey and the MUSYC photometric catalog. The HETDEX spectroscopic data could provide improved distance estimates for many potential galaxy clusters, and combined with modern clustering algorithms, such as Ordering Points To Identify the Clustering Structure (OPTICS; \citealt{Ankerst1999}) or a hierarchical version of DBSCAN (HDBSCAN; \citealt{Campello2013}), which are designed to address the weaknesses of DBSCAN, could provide important insights into the optimization of cluster finding at high redshift. Such advanced clustering searches can also be extended to the large sky-area of PanSTARRS to search for dwarf galaxies or other stellar associations and will eventually be useful to the large-area survey of WFIRST-AFTA. Ultimately, such a study could aid in the development and application of data agnostic machine learning to astronomy, provide improved cluster detection at all redshifts, and suggest potential followup targets requiring the \emph{James Webb Space Telescope}.

\subsection{Outstanding Questions}
The work presented here is only a small piece in the effort to understand how a blind spectroscopic survey could impact the role of galaxy clusters as cosmological probes. As with any study, outstanding questions remain. Here are two potential followup investigations:

\begin{itemize}
	\item A deep study and optimization of the machine learning methods used to predict cluster masses. There are many machine learning algorithms which are capable of combining many cluster observables to predict a cluster mass. In addition, each method has a series of hyper-parameters which can greatly impact the method's performance from one data set to another, and from one potential problem to another. A deep investigation of an appropriate way to combine observational parameters, which parameters to combine, could provide an enormous amount of insight into the best follow up observations, and greatly improve predicted cluster masses. \cite{Acquaviva2016} conducted a similar study for the estimation of galaxy metallicity. A similar study for HETDEX applications would be extremely valuable.
	\item Impact of cluster contamination on the machine learning predicted masses. In our simulated HETDEX survey, we assume that we have knowledge of a cluster's member galaxies, and so this initial work is not effected by the presence of interloping galaxies. A study of how the sparse spatial sampling, and the presence of interloping galaxies \citeeg{Ntampaka2015a} could be mitigated by machine learning, or another method, is extremely important. Such a study will be required to estimate the best possible cluster masses when HETDEX begins taking data. 
\end{itemize} 

Of course, there is also the deep study of the physics surrounding the clusters themselves, through the study of the ICM \citeeg{Mantz2015}, any active galactic nuclei discovered \citeeg{Kirk2015, Sobral2015}, correlations with other observations \citeeg{Pipino2010}, tests for substructure \citeeg{Yu2015}, and many others. 

We provide all of the code used to conduct this study at https://github.com/boada. Large file size prevents including the source data with the analysis routines. The authors are happy to provide them, if requested.
